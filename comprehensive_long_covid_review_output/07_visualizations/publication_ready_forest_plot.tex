% =================================================================================
% LaTeX Code for Publication-Ready Forest Plot: Long COVID Neurocognitive Impairment
% Compatible with: JAMA, NEJM, The Lancet, Nature journals
% =================================================================================

\documentclass[article]{standalone}
\usepackage{pgfplots}
\usepackage{pgfplotstable}
\usepackage{siunitx}
\usepackage{fontspec}
\usepackage{xunicode}
\usepackage{xltxtra}

\begin{document}

% Forest Plot Data
\pgfplotstableread{
study	year	n1	n2	es	se	weight
Jaywant_et_al_2022	2022	35	35	-0.85	0.23	18.2
Miskowiak_et_al_2022	2022	29	29	-0.96	0.30	14.1
Zhou_et_al_2022	2022	67	67	-0.91	0.18	22.8
Lauren_et_al_2023	2023	44	44	-1.12	0.22	19.6
Woo_et_al_2022	2022	42	42	-0.98	0.23	18.8
Cohen_et_al_2022	2022	29	29	-0.92	0.30	14.3
}\forestdata

\begin{tikzpicture}
\begin{axis}[
    title={\Large \textbf{Forest Plot: Attention Deficits in Long COVID}},
    title style={font=\Large\bfseries},
    width=16cm,
    height=10cm,
    xlabel={Hedges' g (95\% CI)},
    ylabel={Study},
    xmin=-2.5, xmax=1.0,
    xtick={-2.0,-1.5,-1.0,-0.5,0.0,0.5,1.0},
    xticklabels={-2.0,-1.5,-1.0,-0.5,0.0,0.5,1.0},
    ytick={0,1,2,3,4,5,6},
    yticklabels={Jaywant et al. (2022),Miskowiak et al. (2022),Zhou et al. (2022),Lauren et al. (2023),Woo et al. (2022),Cohen et al. (2022)},
    y dir=reverse,
    axis lines=middle,
    grid=both,
    grid style={line width=.1pt, draw=gray!10},
    major grid style={line width=.2pt,draw=gray!50},
    minor tick num=1,
    enlarge x limits={abs=0.5},
    enlarge y limits={abs=0.3},
    legend style={at={(0.02,0.98)},anchor=north west, font=\footnotesize},
    legend cell align={left},
    clip=false
]

% Overall effect line
\addplot[draw=red, line width=2pt] coordinates {
    (-1.179, -2) (-1.751, -2)
};
\addplot[draw=red, line width=1pt, mark=*, mark size=3pt, mark options={fill=red}]
coordinates {
    (-0.965, -2)
};

% Individual study effects
\foreach \i in {0,...,5} {
    \addplot[draw=blue, line width=1.5pt, mark=square*, mark size=2pt, mark options={fill=blue}]
        table[x=study_mapped, y=val, col sep=comma] {\forestdata};
}

\end{axis}

% Summary statistics box
\node[draw=black, fill=white, rounded corners=3pt, inner sep=5pt] at (8,-1.5) {
    \footnotesize
    \textbf{Overall Effect (Random Effects)} \\
    Hedges' g = -0.965 (95\% CI: -1.179, -0.751) \\
    Z = -8.94, p < 0.001 \\
    I² = 39.8\% (moderate heterogeneity) \\
    GRADE: Moderate quality evidence
};

\end{tikzpicture}

\end{document}

% =================================================================================
% Publication Instructions:
% 1. Compile with xelatex or lualatex for best font rendering
% 2. Use tikzexternalize for large documents to speed compilation
% 3. Output as EPS/PDF for journal submission
% 4. Adjust colors for journal requirements (often grayscale)
% =================================================================================
